\chapter{ChatScript: Grundlagen}
\label{sec:ChatScript: Grundlagen}

Bei ChatScript handelt es sich um ein Framework, der es Entwicklern erlaubt, regelbasierte Chatbots zu entwickeln. Es handelt sich hierbei um "Natural Language tool" mit eigenem Interpreter, welcher Code in C++ übersezt. 


 

\section{ChatScript: Regeln}
\label{sec:ChatScript: Regeln}

\section{ChatScript: Topics}
\label{sec:ChatScript: Topics}

Die Quelldateien in einem ChatScript-Programm werden auch als "Topics" bezeichnet und haben die Dateiendung .top. Topics können als Zusammenfassung mehrerer Regeln betrachtet werden. Es bietet sich an, Topics als Module eines Chatbots zu handhaben. So wird z.B. in diesem Projekt "Augusta" die Datenbankabfrage als eignenes Topic gehandhabt. 

\section{ChatScript: Konzepte}
\label{sec:ChatScript: Konzepte}

\section{ChatScript: Bedingungen zur Steuerung von Programmablauf}
\label{sec:ChatScript: Bedingungen zur Steuerung von Programmablauf}

\section{ChatScript: Befehle zur Steuerung von Programmablauf}
\label{sec:ChatScript: Befehle zur Steuerung von Programmablauf}

\section{ChatScript: Pattern-Matching}
\label{sec:ChatScript: Pattern-Matching}

\section{ChatScript: Zufällige Ausgabe}
\label{sec:ChatScript: Zufällige Ausgabe}

\chapter{Augusta: Programmlogik}
\label{sec:Augusta: Programmlogik}



Hier, stelle in z.B. einem Automaten dar, wie der Programmablauf normalerweise ablaufen soll
\section{Befehle und Programmablauf}
\label{sec:Befehle und Programmablauf}

\section{Variablen und Programmablauf}
\label{sec:Variablen und Programmablauf}

\section{Konzepte und Programmablauf}
\label{sec:Konzepte und Programmablauf}

\section{Programmablauf im Detail}
\label{sec:Programmablauf im Detail}

