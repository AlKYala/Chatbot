\chapter{Einleitung}
\label{sec:Einleitung}

"`Augusta"' ist ein Chatbot, dessen Aufgabe es ist, dem Kunden Produkte aus dem Uni-Shop zu empfehlen. Dafür muss der Nutzer dem Chatbot via Texteingabe angeben, was gesucht wird. Diese Nutzereingaben werden mit Hilfe von Pattern-Matching analysiert und via PostgreSQL-Schnittstelle werden Antworten basierend auf Queries generiert, die dem geäußerten Wunsch des Nutzers entsprechen.\\
Das Ergebnis des Projekts kann auf GitHub unter \url{https://github.com/AlKYala/Chatbot} eingesehen werden.\\


\section{Ziele des Projekts}
\label{sec:ZielDesProjekts}

Vorrangiges Ziel des Projekts war der Versuch der Entwicklung eines computerlinguistischen Programms. Das Umsetzen von im Studium angeeigneter Fähigkeiten theoretischer und praktischer Natur, wie beispielsweise der Automatentheorie, Datenbankentechnologie und Softwaretechnik, hat eine essentielle Rolle in der Konzeption und Realisierung des Programms gespielt. Weiteres Ziel war das Einarbeiten in ein fremdes Framework mit eigenem Interpreter, hier ChatScript \citep{chatscript2019} von Bruce Wilcox, um ein entsprechendes Programm zu entwickeln.\\


\section{Der Chatbot Augusta}
\label{sec:Augusta}

Im Projektseminar der Computerlinguistik haben wir in kleinen Gruppen jeweils einen Chatbot für eine spezifische Anwendung entwickeln wollen. Unsere Gruppe, bestehend aus Ali Yalama und Julia Roegner, hat sich dazu entschieden, einen Chatbot als Kaufberatung für den Uni-Shop der Universität Trier zu bauen. Der Bot sollte für das Deutsche gelten und eine ausschließlich beratende Funktion haben, das bedeutet, dass der Bot Bestellungen oder Fragen zum Uni-Shop nicht beantworten können muss.\\
Als Basis dienten die Beispieldialoge (s. Anhang), die auch am Ende dieser Dokumentation abgedruckt sind. Hier wird davon ausgegangen, dass der Nutzer noch nicht weiß, welches Produkt er haben möchte, sondern ihm nur bekannt ist, wofür er es möchte, also entweder eine Geschenkidee hat oder etwas für eine bestimmte Anwendung sucht. Dann bietet der Chatbot dem Nutzer die Kategorien an, die in der Datenbank für Anwendungsgebiet oder Geschenkidee hinterlegt sind. Im Weiteren kann der Kunde dem Chatbot noch weitere Informationen über den gesuchten Artikel wie eine Farbe oder den Preis übergeben, am besten in Form eines kompakten Satzes, den der Chatbot analysiert. Dann nutzt der Bot die extrahierten Stichwörter, um in einer Datenbank, in der die Informationen über die Produkte des Uni-Shops gespeichert sind, nach etwas Passendem zu suchen.\\
Der gefundene Gegenstand wird dem Kunden angeboten und der Kunde kann entscheiden, ob er ihn ablehnt, dann sucht der Bot erneut, oder ihn annimmt, woraufhin der Chatbot fragt, ob eine neue Suche mit anderen Kriterien durchführen soll. Wenn nicht, beendet sich der Bot.\\
Der Name des Bots, Augusta, leitet sich vom Namen der antiken Stadt \textit{Augusta Treveorum} an der Mosel her, aus der das heutige Trier hervorgegangen ist.\\


\section{Struktur der Dokumentation}
\label{sec:Inhaltsbeschreibung} 
In Kapitel \ref{sec:InstallationAugusta} werden die einzelnen Versionen Augustas erläutert sowie die Installation beschrieben. In Kapitel \ref{sec:ChatScript: Grundlagen} folgen dann einige grundlegende Regeln zum Schreiben von Chatbots in ChatScript, woran sich ein Kapitel zu der Programmlogik hinter unserem Chatbot anschließt, das Augusta als deterministischen Automaten vorstellt. Informationen zu der Datenbank \textit{Uni-Shop}, in der die Produktinformationen der Artikel gespeichert sind, und ihrer Einbindung sowie der Suchanfragen finden sich in Kapitel \ref{sec:DBuEinbindung}. Auf die Benutzerschnittstelle wird in Kapitel \ref{sec:Benutzerschnittstelle} eingegangen. In Kapitel \ref{sec:Probleme} werden schließlich einige Probleme erläutert, auf die wir gestoßen sind und im Rahmen des Projekts leider nicht mehr lösen konnten.\\
Am Ende der Dokumentation sind die Dialoge abgedruckt, die als Basis dienten, und die Zusammenfassung des Ergebnisses aus einem kleinen Usability Test.\\
Im Literaturverzeichnis finden sich zudem die Foreneinträge, die wir während der Arbeit an ChatScript verfasst haben.
