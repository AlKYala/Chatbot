\chapter{Einleitung}
\label{sec:Einleitung}

Chatbot "Augusta" ist ein Chatbot, dessen Aufgabe es ist, seinem Nutzer Produkte aus dem Uni-Shop zu empfehlen. Dafür muss der Nutzer dem Bot "Augusta" via Texteingabe angeben, was gesucht wird. Diese Nutzereingaben werden mit Hilfe von Pattern-Matching analysiert und via PostgreSQL-Schnittstelle werden Antworten basierend auf Queries generiert, die dem geäußerten Wunsch des Nutzers entsprechen. 

\section{Ziele des Projekts}
\label{sec:ZielDerArbeit}

Vorrangiges Ziel des Projekts war der Versuch der Entwicklung eines computerlinguistischen Programms. Das Umsetzen im Studium angeeigneter Fähigkeiten theoretischer und praktischer Natur, wie z.B. Automatentheorie, Softwareentwurf und Datenbankentechnologie hat eine essentielle Rolle in der Konzeption und Realisierung des Programms gespielt. Weiteres Ziel war das Einarbeiten in ein fremdes Framework mit eigenem Interpreter, hier ChatScript \citep{chatscript2019} von Bruce Wilcox, um ein entsprechendes Programm zu entwickeln. 

\section{Struktur der Arbeit}
\label{sec:Inhaltsbeschreibung} 

