\chapter{Einleitung}
\label{sec:Einleitung}

"`Augusta"' ist ein Chatbot, dessen Aufgabe es ist, seinem Nutzer Produkte aus dem Uni-Shop zu empfehlen. Dafür muss der Nutzer dem Bot "Augusta" via Texteingabe angeben, was gesucht wird. Diese Nutzereingaben werden mit Hilfe von Pattern-Matching analysiert und via PostgreSQL-Schnittstelle werden Antworten basierend auf Queries generiert, die dem geäußerten Wunsch des Nutzers entsprechen. 


\section{Ziele des Projekts}
\label{sec:ZielDesProjekts}

Vorrangiges Ziel des Projekts war der Versuch der Entwicklung eines computerlinguistischen Programms. Das Umsetzen von im Studium angeeigneter Fähigkeiten theoretischer und praktischer Natur, wie beispielsweise der Automatentheorie, Softwaretechnik und Datenbankentechnologie, hat eine essentielle Rolle in der Konzeption und Realisierung des Programms gespielt. Weiteres Ziel war das Einarbeiten in ein fremdes Framework mit eigenem Interpreter, hier ChatScript \citep{chatscript2019} von Bruce Wilcox, um ein entsprechendes Programm zu entwickeln.\\


\section{Der Chatbot Augusta}
\label{sec:Augusta}

Im Projektseminar der Computerlinguistik haben wir in kleinen Gruppen jeweils einen Chatbot für eine spezifische Anwendung entwickeln wollen. Unsere Gruppe, bestehend aus Ali Yalama und Julia Roegner, hat sich dazu entschieden, einen Chatbot als Kaufberatung für den Uni-Shop der Universität Trier zu bauen. Der Bot sollte für das Deutsche gelten und eine ausschließlich beratende Funktion haben, d. h. Bestellungen oder Fragen zum Uni-Shop muss der Bot nicht beantworten können.\\
Als Basis diente der \textcolor[rgb]{1,0,0}{Beispiel-Dialog}, der auch am Ende dieser Dokumentation abgedruckt ist. Hier wird davon ausgegangen, dass der Nutzer noch nicht weiß, welches Produkt er haben möchte, sondern ihm nur bekannt ist, wofür er es möchte, also entweder eine Geschenkidee hat oder etwas für eine bestimmte Anwendung sucht. Dann bietet der Bot dem Nutzer die Kategorien an, die in der Datenbank für Anwendungsgebiet oder Geschenkidee hinterlegt sind. Im Weiteren kann der Kunde dem Chatbot noch weitere Informationen über den gesuchten Artikel wie eine Farbe oder den Preis übergeben, am besten in Form eines kompakten Satzes, den der Chatbot analysiert. Dann nutzt der Bot die extrahierten Stichwörter, um in einer Datenbank, in der die Informationen über die Produkte des Uni-Shops gespeichert sind, nach etwas Passenden zu suchen.\\
Der gefundene Gegenstand wird dem Kunden angeboten und der Kunde kann entscheiden, ob er ihn ablehnt, dann such der Bot erneut, oder annimmt, woraufhin der Bot fragt, ob eine neue Suche mit anderen Kriterien durchführen soll. Wenn nicht, beendet sich der Bot.\\
Der Name des Bots, Augusta, leitet sich vom Namen der antiken Stadt \textit{Augusta Treveorum} an der Mosel her, aus der das heutige Trier hervorgegangen ist.\\


\section{Struktur der Arbeit}
\label{sec:Inhaltsbeschreibung} 

