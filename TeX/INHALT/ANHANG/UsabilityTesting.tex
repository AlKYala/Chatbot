\chapter{Ergebnis des Usability Testing}
\label{sec:UsabilityTesting}

Anfang Juni haben wir unser Programm einer kleiner Evaluierung unterzogen. Die Testgruppe bestand aus zwei Personen, die ein paar Mal mit Augusta gesprochen haben. Im Folgenden findet sich die Zusammenfassung dieses kleinen Usability Testings.\\
Ein Problem dabei war allerdings, dass das Programm zu diesem Zeitpunkt noch einige Bugs aufwies und nicht voll funktionsfähig war. Doch ein kurzes Feedback war möglich. Getestet wurde zudem nur eine Version auf der Konsole.
\vspace{10pt}


Zu Beginn gibt zwei mal in anderer Formulierung die Frage, ob der Kunde hier sei, um etwas zu kaufen (aus INTRO). Diese Fragen, vor allem in doppelter Ausführung, wurden nicht nur als unnötig sondern auch als nervig empfunden. Ebenso wird die Frage angezeigt, wenn der Kunde nach einem Fehler bei der Suche neu anfangen muss. Auch dies war für die Benutzer störend. \\
Verwirrend erschien des Weiteren die Angabe der Kategorien aus Geschenkidee (FIRSTQ). So sei bei "`Geschenk, also zum Beispiel zum Geburtstag oder zu Weihnachten. Gastgeschenk. Mitbringsel. Erinnerungsstück."' nicht klar, dass "`Geschenk"', "`Gastgeschenk"', "`Mitbringsel"' und "`Erinnerungsstück"' um die Auswahlmöglichkeiten handle und es wurde angenommen, dass man auch "`zum Geburtstag"' oder "`zu Weihnachten"' eingeben könne. Generell führte die Ambiguität des Begriffs "`Geschenk"' verwirrend zu Unklarheit.\\
Kritisch betrachtet wurde auch, dass Augusta "`Gerne."' als Ersatz für "`Ja."' nicht akzeptiere.\\
Außerdem zeigten sich die Benutzer von den vielen Eingabemöglichkeiten in KeyExOnesentence beim Erfragen von zusätzlichen Produktkriterien überfordert. So wusste beispielsweise eine Testperson nicht reicht, "`wie genau man das eingeben soll."' So sei unklar gewesen, ob man nur die Begriffe oder einem vollständigen Satz eingeben solle. Ein zweiter Test mit KeyExProdukteigenschaften ist als einfacher empfunden worden, da hier der Bot den Benutzer mit gezielten Fragen nach Produktname, Ausführung und Preis führe.
\vspace{10pt}


Als Reaktion auf die Tests haben wir die doppelte Frage nach der Kaufabsicht aus INTRO herausgenommen. Inzwischen geht Augusta von Anfang an davon aus, dass der Kunde etwas kaufen möchte oder zumindest nach etwas sucht.\\
Die Ausgabe der Kategorien haben wir vor Ort noch in "`Geschenk, Gastgeschenk, Erinnerungsstück und Mitbringsel"' umformuliert. Zudem fragt der Chatbot am Anfang von FIRSTQ ob der Kunde "`etwas zum Verschenken oder für einen besonderen Zweck"' suche, wodurch der Begriff "`Geschenk"' desambiguiert werden soll. Die Testpersonen reagierten auf diese beiden Änderungen sehr positiv. Es sei nun verständlicher. \\
Ursprünglich war KeyExProdukteigenschaften nur als Übergangslösung bis zur Fertigstellung von KeyExOnesentence gedacht. Allerdings haben wir uns dazu entschieden, aufgrund der Bemerkungen der Testpersonen, einen zweiten Chatbot mit KeyExProdukteigenschaften anstatt KeyExOnesentence als Alternative zur Verfügung zu stellen.

